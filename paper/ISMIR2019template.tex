% -----------------------------------------------
% Template for ISMIR Papers
% 2019 version, based on previous ISMIR templates

% Requirements :
% * 6+n page length maximum
% * 4MB maximum file size
% * Copyright note must appear in the bottom left corner of first page
% * Clearer statement about citing own work in anonymized submission
% (see conference website for additional details)
% -----------------------------------------------

\documentclass{article}
\usepackage[T1]{fontenc} % add special characters (e.g., umlaute)
\usepackage[utf8]{inputenc} % set utf-8 as default input encoding
\usepackage{ismir,amsmath,cite,url}
\usepackage{graphicx}
\usepackage{color}

% Optional: To use hyperref, uncomment the following.
% \usepackage[bookmarks=false,hidelinks]{hyperref}
% Mind the bookmarks=false option; bookmarks are incompatible with ismir.sty.

\newcommand{\setNameUpper}{FOURIER}
\newcommand{\setName}{Fourier}

% Title.
% ------
\title{The \setNameUpper~Set: Beats, Downbeats, and Functional Segment Annotations of Western Popular Music}

% Note: Please do NOT use \thanks or a \footnote in any of the author markup

% Single address
% To use with only one author or several with the same address
% ---------------
%\oneauthor
% {Names should be omitted for double-blind reviewing}
% {Affiliations should be omitted for double-blind reviewing}

% Two addresses
% --------------
%\twoauthors
%  {First author} {School \\ Department}
%  {Second author} {Company \\ Address}

%% To make customize author list in Creative Common license, uncomment and customize the next line
%  \def\authorname{First Author, Second Author}


% Three addresses
% --------------
\threeauthors
  {First Author} {Affiliation1 \\ {\tt author1@ismir.edu}}
  {Second Author} {\bf Retain these fake authors in\\\bf submission to preserve the formatting}
  {Third Author} {Affiliation3 \\ {\tt author3@ismir.edu}}

%% To make customize author list in Creative Common license, uncomment and customize the next line
%  \def\authorname{First Author, Second Author, Third Author}

% Four or more addresses
% OR alternative format for large number of co-authors
% ------------
%\multauthor
%{First author$^1$ \hspace{1cm} Second author$^1$ \hspace{1cm} Third author$^2$} { \bfseries{Fourth author$^3$ \hspace{1cm} Fifth author$^2$ \hspace{1cm} Sixth author$^1$}\\
%  $^1$ Department of Computer Science, University , Country\\
%$^2$ International Laboratories, City, Country\\
%$^3$  Company, Address\\
%{\tt\small CorrespondenceAuthor@ismir.edu, PossibleOtherAuthor@ismir.edu}
%}
%\def\authorname{First author, Second author, Third author, Fourth author, Fifth author, Sixth author}


\sloppy % please retain sloppy command for improved formatting

\begin{document}

%
\maketitle
%
\begin{abstract}
    We introduce the \setName\footnote{Temporary name for the blind review process.}~set: a collection of annotations of beats, downbeats, and functional segmentation for over 700 tracks that covers a wide range of western popular music.
    Given the variety of annotated music information types in this set, and how strongly these three types of data are typically intertwined, we believe it could potentially foster research that focuses on multiple retrieval tasks at once.
    The dataset includes additional metadata such as MusicBrainz identifiers to support the linking of the dataset to third-party information or audio data when available.
    We describe the methodology employed in acquiring this set, including the annotation process and song selection. 
    In addition an initial data exploration of the annotations and actual dataset content is conducted. 
    Finally, we provide a series of baselines of the \setName~set with standard beat-trackers, downbeat predictors, and structural segmentation algorithms in the literature.
\end{abstract}
%
\section{Introduction}\label{sec:introduction}

The tasks of beat detection, downbeat prediction, and structural segmentation~\cite{Paulus2010} constitute a fundamental part of the field of MIR.
These three musical characteristics are often related: downbeats define the first beat of a given music measure, and long structural music segments tend to begin and end on specific beat locations --frequently on downbeats.
The automatic estimation of such information could result in better musical systems such as more accurate automatic DJ-ing, better intra- and inter-song navigation, further musicological insights of large collections, \emph{etc}.
While several approaches exploiting more than one of these musical traits have been proposed~\cite{Bock2016, Fuentes2019}, the lack of human annotated data containing the three of them for a single collection are scarce.
This limits the amount of potential of certain methods, especially those that require large amounts of information (e.g., deep learning~\cite{Humphrey2012}).

In this paper we present the \setName~set: human annotations of beats, downbeats, and functional segmentation for over 700 tracks of western popular music.
These annotations were gathered with the aim of having a significant amount of data to train models to improve the prediction of such musical attributes, which would later be applied to various products offered by the company \setName.
By releasing this set to the public, it is our aim to let the research community explore and exploit these annotations to advance the tasks of beat tracking, downbeat prediction, and automatic functional structural segmentation.
We discuss the methodology to acquire these data, including the song selection process, and the inclusion of standard identifiers (Acoustid and MusicBrainz) and onsets for the first 30 seconds of the tracks to allow for other researchers to more easily access and align, when needed, the actual audio content.
Furthermore, we present a series of results with standard algorithmic approaches in the literature with the goal of having an initial public benchmark of this set.

The rest of this work is organized as follows: Section~\ref{sec:background} contains a review of the most relevant public datasets of the tasks at hand; Section~\ref{sec:dataset} discusses the actual \setName~set, including the data gathering, their formatting, and various statistics; Section~\ref{sec:results} presents various benchmarks in the set; and Section~\ref{sec:conclusions} draws some final conclusions and discloses future remarks.
%
\section{Background}\label{sec:background}

In this section we discuss previously published datasets and the tasks they aim to solve.

% Introduce on beat annotation
\subsection{Beat and Downbeat Tracking Sets}

Over the last 15 years, many annotated datasets for beat and downbeat tracking have appeared in the literature whose primary purpose has been to allow the comparison of newly proposed and existing algorithms. However, the well-known difficulties of sharing the audio component of large annotated datasets has led to a rather ad-hoc usage of different datasets within the literature, and to a lesser extent, the choice of which evaluation metrics are selected to report accuracy. Conversely, the MIREX evaluation campaign provides a more rigid model for evaluation, by withholding access to private test datasets, and instead relying on the submission of the competing algorithms in order to compare them under controlled conditions. To this end, MIREX can be a useful reference point to consider these two music analysis tasks from the perspective of annotated data.  

The MIREX Audio Beat Tracking (ABT) task\footnote{https://www.music-ir.org/mirex/wiki/2006:Audio\_Beat\_Tracking} first appeared in 2006 and ran on a single dataset \cite{moelants04icmpc,mckinney07jnmr} with the performance of the submitted algorithms determined using one evaluation metric, the P-Score. After a brief hiatus, the task reappeared in 2009 with the addition of a dataset of Chopin Mazurkas \cite{sapp07ismir}, and the inclusion of multiple evaluation metrics \cite{davies09techreport}. The task continued to run in this way until the incorporation of the SMC dataset \cite{holzapfel12taslp} in 2012, from which point it has remained constant. In 2014, the Audio Downbeat Estimation (ADE) task\footnote{https://www.music-ir.org/mirex/wiki/2014:Audio\_Downbeat\_Estimation} was launched which comprised six different datasets from diverse geographic and stylistic sources: The Beatles \cite{Mauch2009a}; Hardcore, Jungle, Drum and Bass (HJDB) \cite{hockman12ismir}; Turkish \cite{srinivasamurthy14jnmr}; Ballroom \cite{krebs13ismir}; Carnatic \cite{srinivasamurthy14icassp}; and Cretan \cite{holzapfel14ismir}, with the evaluation conducted using the F-measure. While the datasets contained with these two MIREX tasks are by no means exhaustive, they provide a useful window to explore both how the audio data is chosen and how the annotation is conducted for these MIR tasks. To this end, we provide the following breakdown of different properties including reference to both MIREX and non-MIREX datasets. 

\textbf{Duration}: Unlike the task of structural segmentation, beat and downbeat tracking datasets can be comprised of musical excerpts \cite{hainsworth04jasp,mckinney07jnmr,hockman12ismir, krebs13ismir} rather than full tracks \cite{Goto2002,Mauch2009a,digiorgi2016jaes,eremenko18ismir}. \textbf{Number of annotators}: The initial MIREX beat tracking dataset \cite{mckinney07jnmr} was unique in that it contained the annotations of $40$ different people who tapped the beat to the music excepts. Conversely, other datasets used multiple annotators contributing across the dataset \cite{holzapfel12taslp}, a single annotator for all excerpts \cite{hainsworth04jasp}, or even deriving the annotations in a semi-automatic way from the output of an algorithm \cite{Mauch2009a}. \textbf{Annotation post-processing}: Given some raw tap times or algorithm output, these can either be left unaltered \cite{mckinney07jnmr} or, as is more common, iteratively adjusted until they are considered perceptually accurate by the annotator(s)~\cite{holzapfel12taslp, hainsworth04jasp, hockman12ismir}. \textbf{Style-specificity}: While some datasets are designed to have broad coverage across a range of musical styles \cite{Goto2002,marchand15dafx,hainsworth04jasp}, others target a particular group of styles \cite{hockman12ismir,krebs13ismir}, a single style \cite{eremenko18ismir}, the work of a given artist \cite{Mauch2009a,digiorgi2016jaes} or even multiple versions of the same pieces \cite{sapp07ismir}. \textbf{Western / Non-Western}: Similarly, the make up of the dataset can target underrepresented non-western music \cite{srinivasamurthy14icassp, srinivasamurthy14jnmr,nunes15ismir}. \textbf{Perceived difficulty}: Finally, the choice of musical material can be based upon the perceived difficulty of the musical excerpts, either from the perspective of musical or signal level properties \cite{holzapfel12taslp}. 


\subsection{Structural Segmentation Sets}

The task of structural segmentation has been particularly active in the MIR community since the late 2000s.
Similarly to the beat tracking task, several datasets have been published, and some of them have evolved over time.
One of the major challenges with structural segmentation is that this task is regarded as both ambiguous (i.e., there may be more than one valid annotation for a given track~\cite{McFee2017}) and subjective (i.e., two different listeners might perceive different sets of segment boundaries~\cite{Bruderer2009}).
This has led to different methodologies when annotating and gathering structural datasets, thus having a diverse ecosystem of sets to choose from when evaluating automatic approaches.

The first time this task appeared on MIREX was in 2009,\footnote{https://www.music-ir.org/mirex/wiki/2009:Structural\_Segmentation} where annotations from The Beatles dataset (included in the set described above) and a subset of the Real World Computing Popular Music Database (RWC)~\cite{Goto2002} were employed.
These sets contain several functional segment annotations for western (The Beatles) and Japanese (RWC) popular music.
These segment functions describe the \emph{purpose} of the segments, e.g.: ``solo'', ``verse'', ``chorus.''
A single annotation per track is provided for these two sets.
The Beatles dataset was further revised at the Tampere University of Technology,\footnote{http://www.cs.tut.fi/sgn/arg/paulus/beatles\_sections\_TUT.zip} and additional functional segment annotations for other other bands were added to The Beatles set, which became known as the Isophonics Dataset~\cite{Mauch2009a}.
No beat or downbeat annotations were provided to the rest of the tracks in Isophonics, and the final number of tracks with functional structural segment annotations is 300.
The number of annotated tracks in the RWC is 365.

To address the open problems of ambiguity and subjectivity, further annotations per track from several experts could be gathered.
That is the case with The Structural Annotations for Large Amounts of Music Information (SALAMI) dataset, where most of its N tracks have been annotated by at least 2 humans.

Since these challenges remain open problems, and structural segmentation remains 

, when it first appeared in MIREX.
TUT Beatles (corrections)

Segmentation datasets: SALAMI~\cite{Smith2011}, Real World Computing Popular Music Database (RWC)~\cite{Goto2002}, Mazurka\footnote{http://mazurka.org.uk}~\cite{4959972}, SPAM~\cite{Nieto2016}

Combined datasets: The Beatles Dataset.

\section{The \setName~Set}\label{sec:dataset}

Explanation of dataset.\footnote{URL not displayed for the blind review process.}

\subsection{Data Gathering}

How were the songs selected. Show genre distribution plot.

Who annotated them (e.g., professional musicians?)

How they annotated them (tools, revisions, etc.). Why only flat segmentation.

Why were the songs selected (motivation behind \setName).

Additional data could be obtained by data augmentation~\cite{Mcfee2015}.

\subsection{Data Formatting}

JAMS~\cite{Humphrey2014}.

MusicBrainz IDs.

Acoustid IDs.

Train/Test splits.

\subsection{Data Statistics}

Show plots and statistics for: 

\subsubsection{beats}

\subsubsection{downbeats}

\subsubsection{segmentation}

How many segments start/end on downbeats?

\section{Results}\label{sec:results}

\subsection{Beat Results}

Beat results

\subsection{Downbeat Results}

Downbeat results

\subsection{Segmentation Results}

Segmentation results. Using~\cite{Nieto2016}. Serr\`a algorithm~\cite{Serra2014}. Results included in CVS format for all dataset.

\begin{figure}
    \centerline{\includegraphics[width=\columnwidth]{figs/segment_results.pdf}}
    \caption{Segmentation results}
    \label{fig:segment_results}
\end{figure}

\subsection{Beat + Segmentation Results}

Beat + Segmentation.

\section{Conclusions}\label{sec:conclusions}

Potential of combined data + large amounts of annotations.

% For bibtex users:
\bibliography{ISMIRtemplate,beat_data_refs}

\end{document}
